%Preamble
\documentclass[12pt,a4paper]{article}        %Define text type and basic formatting
\usepackage[document]{ragged2e}   %Text alignment https://www.overleaf.com/learn/latex/Text_alignment

\usepackage[utf8]{inputenc}     %Usage of UTF-8 for umlauts
\usepackage[ngerman]{babel}     %Paper language;
\pagenumbering{arabic}  % "Normal" page numbering
%Set line spacing to 1.5
% \usepackage{setspace}
% \onehalfspacing

%Citation and reference
\usepackage[backend=biber,
  style=authoryear,
  citestyle=authoryear-comp,
  hyperref=true,
  giveninits,    %Shorten first names to initial
  uniquename=init,    %prevent name disambiguation
  sorting=nyt,  % sort by name, year, title
  natbib,  %enable citep/citet(parentheses only around year)
  maxbibnames=99,  %show all names in bibliography (no influence on in-text citation)
  minbibnames=1  %show at least one name before et al
]{biblatex}   %REFERENCES https://www.overleaf.com/learn/latex/Bibliography_management_in_LaTeX
\addbibresource{references.bib}     %Lib file
\usepackage[nottoc,numbib]{tocbibind}   %add bibliography to toc

%page citation in text with colon: https://tex.stackexchange.com/questions/433122/changing-comma-in-textcite-to-colon
%\DeclareFieldFormat{postnote}{#1}
%\DeclareFieldFormat{multipostnote}{#1}
%\renewcommand\postnotedelim{\addcolon\addspace}
\renewcommand\nameyeardelim{\addcomma\space} %comma between author and year
%u.a. as et al.
\DefineBibliographyStrings{german}{%
  andothers = {et al.},
}
%Replace and/und with &
\renewcommand*{\finalnamedelim}{%
  \ifnumgreater{\value{liststop}}{2}{\finalandcomma}{}%
\addspace\&\space}%

%images
\usepackage{graphicx}       %Required for adding images
\graphicspath{{images/}}    %image path
\usepackage{wallpaper}  %Page background img

\usepackage{parskip}        %Prevent indention of paragraphs
\usepackage{mathtools}    %required for math formulas
\usepackage{amssymb}    %mathematical symbols
\usepackage{listings}   %For code listings

%Use markdown in LaTex https://de.overleaf.com/learn/latex/Articles/How_to_write_in_Markdown_on_Overleaf
\usepackage[footnotes,definitionLists,hashEnumerators,smartEllipses,hybrid]{markdown}

\usepackage{titlesec}   %Style titles
\usepackage{fancyhdr}   %Header/footer
%\usepackage[bottom]{footmisc}   %Foot notes, at end of page

%Links
\usepackage[colorlinks,
  pdfpagelabels,
  pdfstartview = FitH,
  bookmarksopen = true,
  bookmarksnumbered = true,
  linkcolor = black,
  plainpages = false,
  hypertexnames = false,
  citecolor = black,
urlcolor = black]{hyperref}   %Hyperref pkg -> clickable links and TOC
\usepackage{csquotes}   % Autostyle quotes language-specific

%Font settings
\renewcommand{\familydefault}{\sfdefault}       %Text sans-serif
\renewcommand{\headrulewidth}{0pt}
\pagestyle{fancy}

%Footer
\fancyhf{}      %Clear all header/footer stylings
% \lfoot{\thedate\hspace{1pt}}
% \cfoot{Proposal Bachelorarbeit\\«Video- und bildbasierte Desinformation auf Social-Media-Plattformen in der Schweiz» (Arbeitstitel)} % Footnote
\rfoot{\thepage\hspace{1pt}}        %Add page number

%Title page settings
\usepackage{pdfpages}
\usepackage{titling}    %Title page styling
\title{«Analyse (audio-) visueller Desinformation auf Social-Media-Plattformen in der Schweiz» (Arbeitstitel)}        %Document Title
\author{Yannick Spriessler}     %Author of paper
\date{\today}     %Date of paper; ALTERNATIVE: \today

% Separate bibliography for images
% \defbibheading{imagecredits}{\section*{Bildverweis}}
% \addbibresource{img-references.bib}

%___________________________________________________________________________________________
%TITLE PAGE
\begin{document}
\begin{titlingpage} %Start titling page
  \includepdf{Titelblatt_Thesis}
  \nocite{howard_trees_2017}  %Citation for title page, only in image bibliography
\end{titlingpage}
\pagebreak      %Insert page break
%-----------------------------------------------
%TOC
\thispagestyle{empty}
\setcounter{page}{0}    %Set page no.
\tableofcontents        %Content index
\pagebreak
%-----------------------------------------------
%DOC
\renewcommand{\abstractname}{Abstract}
\begin{abstract}
  \setlength{\parindent}{0pt}  
    asdfghjk \\
\end{abstract}

\textbf{Keywords:} \textit{Desinformation, Social Media, audiovisuelle Inhaltsanalyse}


\textbf{Zitiervorschlag:}
\linebreak
%\frontmatter   %If foreword
\pagebreak

%\mainmatter    %Main part if foreword used
\section{Einleitung}
Spätestens seit den US-Wahlen 2016 hat der Begriff «Fake News» stark an Bedeutung gewonnen. Zwar handelt es sich dabei nicht um eine Neuerscheinung und falsche Informationen wurden lange vor der Entwicklung des Internets verbreitet \parencites[214]{allcott_social_2017}[247]{hohlfeld_schlechte_2020}[1]{khan_fake_2021}, dennoch ist das Problem unter anderem durch die Verbreitung über das Internet weiter stark angewachsen \parencites[214-215]{allcott_social_2017}[1]{khan_fake_2021}[1]{lazer_science_2018}. 

Die Bachelorarbeit befasst sich damit, welche inhaltlichen und gestalterischen Merkmale audiovisuelle und bildbasierte Desinformation auf Social-Media-Plattformen in der Schweiz aufweist. \linebreak
Ziel der Arbeit ist es zum einen herauszufinden, ob es mögliche Muster und
Strategien hinter der Produktion der Inhalte gibt. Zum anderen soll auch ein
allgemeines Verständnis über die entsprechenden Inhalte gewonnen werden. \linebreak
Die Erkenntnisse der Bachelorarbeit werden anschliessend verwendet, um darauf basierend eine interaktive Aufklärungsplattform zu gestalten.

\subsection{Relevanz}
Gesellschaft -> JAMES-Studie, 

\subsection{Forschungsfrage}
\pagebreak
\section{Forschungsstand}
Definition von FN: \cites{tandoc_defining_2018}{marx_fake_2020}
\pagebreak
\section{Methode}
\subsection{Methodenwahl}
\subsection{Untersuchungszeitraum}
\subsection{Räumlicher Geltungsbereich}
\subsection{Analysegegenstand}
\subsubsection{Methodisches Vorgehen}
\subsubsection{Auswertung}
\pagebreak
\section{Ergebnisse}
\pagebreak
\section{Schlussteil}
\subsection{Diskussion}
\subsection{Beantwortung der Forschungsfrage}
\subsection{Weiterer Forschungsbedarf}

\pagebreak
%-----------------------------------------------
%BIB
\section{Verzeichnisse}
\subsection{Literatur}
\printbibliography[heading=none, nottype=artwork]
 \subsection{Bildverweise}
 \printbibliography[heading=none, type=artwork] %Only reference artworks
 \subsubsection{Abbildungsverzeichnis}
 \listoffigures[heading=none]
\subsection{Hilfsmittelverzeichnis}
\begin{table}[]
\begin{tabular}{lllll}
fb & dfgfn & \\
   &       &  \\
   &       &  \\
   &       & 
\end{tabular}
\end{table}
\section{Anhang}
\end{document}
