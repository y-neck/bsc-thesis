%Preamble
\documentclass[12pt,a4paper]{article}        %Define text type and basic formatting

\usepackage[utf8]{inputenc}     %Usage of UTF-8 for umlauts
\usepackage[ngerman]{babel}     %Paper language;
\pagenumbering{arabic}  % "Normal" page numbering
%Set line spacing to 1.5
\usepackage{setspace}
\onehalfspacing

%Citation and reference
\usepackage[backend=biber,
  style=authoryear,
  citestyle=authoryear-comp,
  hyperref=true,
  giveninits,    %Shorten first names to initial
  uniquename=init,    %prevent name disambiguation
  sorting=nyt,
  natbib  %enable citep/citet(parentheses only around year)
]{biblatex}   %REFERENCES https://www.overleaf.com/learn/latex/Bibliography_management_in_LaTeX
\addbibresource{./references.bib}     %Lib file
\usepackage[nottoc,numbib]{tocbibind}   %add bibliography to toc

%page citation in text with colon: https://tex.stackexchange.com/questions/433122/changing-comma-in-textcite-to-colon
\DeclareFieldFormat{postnote}{#1}
\DeclareFieldFormat{multipostnote}{#1}
\renewcommand\postnotedelim{\addcolon\addspace}
\renewcommand\nameyeardelim{\addcomma\space} %comma between author and year
%u.a. as et al.
\DefineBibliographyStrings{german}{%
  andothers = {et al.},
}
%Replace and/und with &
\renewcommand*{\finalnamedelim}{%
  \ifnumgreater{\value{liststop}}{2}{\finalandcomma}{}%
\addspace\&\space}%

%images
\usepackage{graphicx}       %Required for adding images
\graphicspath{{images/}}    %image path
\usepackage{wallpaper}  %Page background img

\usepackage{parskip}        %Prevent indention of paragraphs
\usepackage{mathtools}    %required for math formulas
\usepackage{amssymb}    %mathematical symbols
\usepackage{listings}   %For code listings

%Use markdown in LaTex https://de.overleaf.com/learn/latex/Articles/How_to_write_in_Markdown_on_Overleaf
\usepackage[footnotes,definitionLists,hashEnumerators,smartEllipses,hybrid]{markdown}

\usepackage{titlesec}   %Style titles
\usepackage{fancyhdr}   %Header/footer
\usepackage[bottom]{footmisc}   %Foot notes, at end of page

%Links
\usepackage[colorlinks,
  pdfpagelabels,
  pdfstartview = FitH,
  bookmarksopen = true,
  bookmarksnumbered = true,
  linkcolor = black,
  plainpages = false,
  hypertexnames = false,
  citecolor = black,
urlcolor = black]{hyperref}   %Hyperref pkg -> clickable links and TOC
\usepackage{csquotes}   % Autostyle quotes language-specific

%Font settings
\renewcommand{\familydefault}{\sfdefault}       %Text sans-serif
\renewcommand{\headrulewidth}{0pt}
\pagestyle{fancy}

%Footer
\fancyhf{}      %Clear all header/footer stylings
\lfoot{\thedate\hspace{1pt}}
\cfoot{Exposé Bachelorarbeit\\«Video- und bildbasierte Desinformation auf Schweizer Social-Media-Plattformen» (Arbeitstitel)}
\rfoot{\thepage\hspace{1pt}}        %Add page number

%Title page settings
\usepackage{titling}    %Title page styling
\title{«Video- und bildbasierte Desinformation auf Schweizer Social-Media-Plattformen» (Arbeitstitel)}        %Document Title
\author{Yannick Spriessler}     %Author of paper
\date{20.12.2024}     %Date of paper; ALTERNATIVE: \today

%___________________________________________________________________________________________
%TITLE PAGE
\begin{document}
\begin{titlingpage} %Start titling page
  \begin{center}
    \begin{large}
      Fachhochschule Graubünden, Institut für Multimedia Production (IMP)\\ %The name of your university
    \end{large}
    \vspace{2cm} %You can control the vertical distance
    \begin{LARGE}
      \textbf{\thetitle} \\
    \end{LARGE}
    \vspace{1cm}
    \begin{large}
      Exposé Bachelorarbeit\\
    \end{large}
    \vspace{5cm} %Put the distance you need.
    Autor: \theauthor \\
    Studiengang: BSc Multimedia Production \\
    Matrikel-Nr.: 22-160-360 \\
    Adresse: Breitenstrasse 26, 4416 Bubendorf \\
    E-Mail: yannick.spriessler@stud.fhgr.ch \\ \\
    Modul: SRSP 5 \\
    Referentin: Prof. Dr. Franziska Oehmer-Pedrazzi \\
    Koreferent: Benjamin Hanimann \\ \\
    Datum: \thedate
  \end{center}
\end{titlingpage}
%\pagebreak      %Insert page break
%-----------------------------------------------
%TOC
% \thispagestyle{empty}
% \setcounter{page}{0}    %Set page no.
% \tableofcontents        %Content index
% \pagebreak
%-----------------------------------------------
%DOC
%\begin{abstract} \end{abstract}    %If abstract is used

%\frontmatter   %If foreword
%\mainmatter    %Main part if foreword used

\section{Einleitung}
\subsection{Einführung ins Thema}
Die Bachelorarbeit wird sich damit befassen, wie Social-Media-Plattformen im DACH-Raum verwendet werden, um durch video- und bildbasierte Desinformation politische oder gesellschaftliche Narrative zu stärken. \\
Inhalte, welche von Faktencheck-Plattformen aus dem DACH-Raum als Desinformation anerkannt wurden, werden durch eine Inhaltsanalyse strukturell untersucht, um daraus Erkenntnisse über oft verwendete Mechanismen zu gewinnen. Diese werden danach verwendet, um darauf basierend als Lehrprojekt eine interaktive Aufklärungsplattform für Nutzende dieser sozialen Plattformen zu gestalten.

\subsection{Relevanz}
\subsubsection{Gesellschaft}
\subsubsection{Wissenschaft}
– Darstellung der Relevanz des Themas (gesellschaftliche, kommunikationspraktische und/oder wissenschaftliche Relevanz)
\section{Bachelorthesis (Umfang ca. 1 Seite)}
\subsection{Forschungsfrage}
– Spezifische Zielsetzung und Fragestellung der Bachelorthesis
\subsection{Forschungsstand}
– Erste Angaben zum Forschungsstand: Auf welche Literatur stützt sich die Bachelorthesis? Welches sind die zentralen Erkenntnisse und Argumente, die die Fragestellung begründen? Welches sind die relevanten Publikationen und Autor:innen in diesem Bereich?
\section{Methodik}
– Geplantes Vorgehen: Methode(n) (Inhaltsanalyse, Befragung, Beobachtung usw.) sowie Forschungsdesign (Analysegegenstand, Analysezeitraum usw.)
\section{Lehrprojekt}
– Inhalt und Idee
\\– Geplante Umsetzung (Webseite, Film, Magazin usw.)
\\– Zielsetzung und Zielgruppe
\pagebreak
\section{Zeitplan}

the quick brown fox jumps over the lazy dog
fgh\smartcite[1-500, 28]{eg34} %like \footcite{\parencite{}}

\[x=something\]
See also \cite{eg34}
\blockquote[Gwgwhe5]{. Lorem ipsum dolor sit amet  bkjkb djblakBSD HNOFClBrI fczubj jbijareg iohioeksfg egkEG IJOlewgnkNERG Kenfsl rjhgbnlnkelnfeklF […]} \parencite[22]{eg34}\\ %line break
f uhe iogoweg oiuhlswer, as explained by \textcite{eg34}

\pagebreak
%-----------------------------------------------
%BIB

\printbibliography[heading=bibintoc, title={Literaturverzeichnis}] %Literaturverzeichnis, auch im TOC
– Einheitliche Auflistung sämtlicher im Exposé direkt und indirekt zitierten Publikationen, empfohlener Zitierstil: APA (siehe «Leitfaden für die Bachelorarbeit») \\
– Evtl. KI-Tools auflisten
\end{document}
